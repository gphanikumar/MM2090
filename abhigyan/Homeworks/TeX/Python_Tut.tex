\documentclass[12pt,a4paper]{article}
\author{Abhigyan Chattopadhyay\\
ME19B001}
\title{Basics of Python}
\usepackage{graphicx}

\usepackage{color}
\definecolor{darkgreen}{rgb}{0,0.5,0}
\definecolor{gray}{rgb}{0.5,0.5,0.5}
\definecolor{mauve}{rgb}{0.58,0,0.82}
\definecolor{lightblue}{rgb}{0.4,0.4,0.9}
\usepackage{listings}
\lstset{
  frame=tb,
  language=Python,
  aboveskip=2mm,
  belowskip=2mm,
  showstringspaces=false,
  columns=flexible,
  basicstyle={\small\ttfamily},
  numbers=none,
  numberstyle=\tiny\color{orange},
  keywordstyle=\color{blue},
  commentstyle=\color{lightblue},
  stringstyle=\color{mauve},
  breaklines=true,
  breakatwhitespace=true,
  tabsize=4
}


\begin{document}
\maketitle
\begin{figure}[h]
\begin{center}
\includegraphics[scale=0.1]{python_img}
\end{center}
\end{figure}

\section{Basic Maths}
\subsection{Operators available:}
\begin{itemize}
\item + (add)
\item - (subtract)
\item * (multiply)
\item / (normal divide)
\item // (floor divide)
\item ** (power)
\item \% (modulo)
\end{itemize}

\begin{lstlisting}
>>> 2+2
4
>>> 50 - 5*6
20
>>> 8 / 5  # normal division always returns a floating point number
1.6
>>> 8 // 5 # floor division always returns an integer
1
>>> 13 % 5 # modulo operator
3
>>> 12**5
248832
\end{lstlisting}

\section{Basic Input and Output}

\end{document}