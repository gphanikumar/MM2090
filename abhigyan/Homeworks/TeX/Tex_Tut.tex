\documentclass[12pt,a4paper]{article}
\usepackage{graphicx}
\usepackage{hyperref}
\usepackage{amsmath}
\usepackage{esint}
\title{Summary of \LaTeX{}}
\author{Abhigyan Chattopadhyay\\
ME19B001}
\setcounter{section}{-1}

\begin{document}
\maketitle
\section{Introduction}
\LaTeX{} is a document preparation system, which results in very high quality text output. Unlike Microsoft Office, or LibreOffice, or any of the document editing suites, \LaTeX{} is not a WYSIWYG (what you see is what you get) editor, and writing a document in \LaTeX{} is more like coding a program. This brings in two things:

\begin{itemize}
\item Uniformity in document looks, and
\item Abstraction and automation of unimportant tasks like text alignment and font sizes
\end{itemize}	

Both of these tasks invariably take up a large amount of our productivity in WYSIWYG editors, hence \LaTeX{} is used a lot in scientific publications and is considered a de-facto standard. Here's the \href{https://www.latex-project.org/}{\LaTeX{}} website.

Basically, \LaTeX{} enables us to create and display the following:

\begin{enumerate}
\item Numbered and Bulleted lists (this is a numbered list)
\item Tables
\item Equations
\item Citations (using bibtex)
\item Hyperlinks (using hyperref)
\item Figures (using graphicx)
\end{enumerate}

\section{Displaying \LaTeX{} Features}

Now, we will draw a table containing some useful Scientific Programming Languages, their website (as a hyperlink), and the year in which they were released:

\begin{tabular}{|l|l|r|}
\hline
Language & Website & Year\\
\hline
Octave & \href{https://gnu.org/software/octave/}{https://gnu.org/software/octave/} & 1988\\
Julia & \href{https://julialang.org/}{https://julialang.org/} & 2012\\
R & \href{https://www.r-project.org/}{https://www.r-project.org/} & 1993\\
Scilab & \href{https://www.scilab.org/}{https://www.scilab.org/} & 1990 \\
Perl & \href{https://www.perl.org/}{https://www.perl.org/} & 1987 \\
Scala & \href{https://scala-lang.org/}{https://scala-lang.org/} & 2004\\
\hline
\end{tabular}

Here, we will show a figure:

\begin{figure}[h]
\begin{center}
	\includegraphics[scale=0.1]{python_img}
\end{center}
\end{figure}

Next, let's write an equation. We'll be writing the Divergence Theorem:

\begin{equation}\label{div}
	\iiint_\mathcal{V} (\nabla \cdot \mathbf{v})d\tau = \oiint_\mathcal{S} \mathbf{v} \cdot d\mathbf{a}
\end{equation}

Now, we're referring to \ref{div} saying that it's also called the Gauss Divergence Theorem.

Now, I'm going to try to cite a paper: \cite{phani2005}.

\bibliography{Biblio}
\bibliographystyle{ieeetr}

\end{document}